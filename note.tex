\documentclass[12pt, a4paper, oneside]{ctexart}
\usepackage{amsmath, amsthm, amssymb, graphicx}

%导言区
\title{课程《信号与系统》笔记}
\author{NH5}
\date{更新于2025.3.8}

\begin{document}
\maketitle

\section{信号和系统的概念与分类}
\subsection{信号的分类}
信号可以从如下四个维度进行分类:

确定与随机、连续与离散、周期与非周期、能量与功率

\subsubsection{确定与随机}
\textbf{确定信号}:具有确定时间函数的信号

\textbf{随机信号}:不具有确定性函数,只能从统计意义上描述

\subsubsection{连续与离散}
连续信号和离散信号的区别在于定义域是否连续,类似函数和数列

\subsubsection{周期与非周期}
含义类似周期函数或周期数列

\textbf{注意:两个周期数列的组合不一定是周期数列}

例如:
\begin{align*}
    f_1(t) &= \cos (t)\\
    f_2(t) &= \cos (0.1 \pi t)\\
    f(t) &= f_1(t) + f_2(t)
\end{align*}

\subsubsection{能量与功率}
信号的能量与功率通过如下公式计算:

能量:
\begin{align*}
    W &= \lim_{T \to \infty}\int_{-T}^{T} |f(t)|^2 dt\\
    W &= \lim_{N \to \infty}\sum_{-N}^{N}|f(k)|^2
\end{align*}

功率:
\begin{align*}
    P &= \lim_{T \to \infty}\frac{1}{2T}\int_{-T}^{T}|f(t)|^2 dt\\
    P &= \lim_{N \to \infty}\frac{1}{2N}\sum_{-N}^{N}|f(k)|^2
\end{align*}

能量信号:
\[
    0<W<\infty,P\to 0
\]

功率信号:
\[
    0<P<\infty,W\to 0
\]

\subsection{系统的分类}

\section{时域分析}
\subsection{信号的时域运算}
\subsection{系统的时域分析}

\section{频域分析}
\subsection{信号的频域分析}
\subsection{系统的频域分析}

\section{复频域分析}
\subsection{信号的复频域分析}
\subsection{系统的复频域分析}

\end{document}